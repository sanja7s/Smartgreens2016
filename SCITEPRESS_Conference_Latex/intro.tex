\section{\uppercase{Introduction}}
\label{sec:introduction}

\noindent 
This paper presents the design of YouPower, a social smart grid platform that is designed as a means to explore the potential and challenges of supporting social participation, awareness and engagement of power gird users (\url{https://app.civisproject.eu}). The goal of developing such a system is to make energy visible, to inform users' energy know-how, to promote
pro-environmental social norms, and to facilitate users in their day-to-day life to take energy-friendly actions together with online communities. 

X The idea of linking smart grids with (online) \textit{Social Network}s (SNs) as a joint R\&D topic has recently caught
much attention in media
\cite{Boslet2010,Chima2011,Erickson2012,Fang2013}. There are many
research efforts on either topics, but research on combining SNs with
smart grids has just started. A number of recent works propose frameworks or
approaches that interconnect smart meters (or smart homes) as SNs for
energy management and sharing \cite{Ciuciu2012,Steinheimer2012}. In
addition, Silva et al.~\cite{Silva2012} conducted surveys to
understand user needs for energy services including SN
services. Several frameworks or simulation models for demand side
management and value-added web services with SN aspects have been
developed~\cite{Chatzidimitriou2013,De-Haan2011,Lei2012}. Others have
used simulation models to demonstrate the feasibility of social
coordination in supply and demand \cite{Skopik2014,Worm2013}.
% 
Our research interest expands on the related work in that it focuses
on smart grid user communities. The research is performed within the
framework of the EU FP7 CIVIS
project (\url{www.civisproject.eu}).

Project Context 
