
\section{\uppercase{Design Concept}}
\label{sect:design_concept}

\subsection{Action Suggestions}
\label{sect:tips}

\begin{figure*}[t!]
\centering
%\frame{\includegraphics[width=0.7\linewidth]{img/brf.pdf}}
\shadowimage[width=.296\linewidth]{img/action2.pdf}
\shadowimage[width=.296\linewidth]{img/action1.pdf}
\shadowimage[width=.298\linewidth]{img/action3.pdf}
\caption{Action suggestion part of YouPower}
\label{fig:action}
\end{figure*}

This part of YouPower (see screenshots in Figure \ref{fig:action}) aims to provide users easy access to practical and inexpensive suggestions (or tips) to (1) increase energy awareness, (2) inform energy know-how, and to (3) shape their long-term behaviors related to household energy consumption.
% 
We collected about 50 suggestions (\url{https://goo.gl/R11QdZ}) from credible sources such as national and international energy agencies and associations. There are routine actions such as ``don't keep hot water flowing when you wash your dishes by hand'', regular actions such as ``defrost your fridge in $x$ days'', and one time actions such as ``install a programmable thermostat''.  
Each action is accompanied with a short explanation that mainly focuses on intrinsic values to target long-term sustainable behaviors, the estimated impact and entailed effort (on a scale of 1 to 5), and the information about how many users are taking the action. 

Users can choose to take a few actions at a time and are suggested with a new action when one is completed. 
Some suggestions can be triggered by time, e.g., ``defrost your fridge in $x$ days.'' In such cases, the app reminds the users of the pending actions they are interested in. 
%A user has an ``action list'' that registers the user's actions. 
%
When an action is completed, the user is awarded with points (displayed as \textit{Leaves}) associated to the effort and impact level of that action. 
A user may also choose to abandon or reschedule an accepted action. 
Upon action completion and cancellation, a user is asked to give feedback. 
The user may ``like'' and ``share'' an action, rate the effort level of the action and give comments. 

%\begin{figure}
%\begin{center}
%	\begin{minipage}[t]{0.44\textwidth}
%	   \frame{\includegraphics[width=\textwidth]{img/action_completed.pdf}}
%	    \caption{The feedback form when a user completes an action.}\label{fig:action_completed}
%	  \end{minipage}
%	  \hfill
%	\begin{minipage}[t]{0.44\textwidth}
%	   \frame{\includegraphics[width=\textwidth]{img/action_not_completed.pdf}}
%	   \caption{The feedback form when a user abandons an action.}\label{fig:action_not_completed}
%	  \end{minipage}
%\end{center}
%\end{figure} 

\paragraph{Engagement in Household and Communities} 
To engage each member in a household, the app allows a user to add members (who are also YouPower users) to his/her household. 
A user can see the actions of household members, and add their actions to his/her own action list.
% 
A user can also join communities and participate in discussions to exchange their ideas and share experiences. The top actions (the ones with most participants) in a community are displayed to members to introduce social norms. 


%This application allows users to create and participate in community and personal energy challenges. It gives participants feedback on their performance during the challenge period and provides encouragement for participation using micro-actions. Users who performed well can share their success stories with other users. The application also allows for friends and group discussion. The community challenges can be incentivised by local investments (solar panels for the building), or through funding projects in developing countries (a school in Uganda). This application tries to motivate user engagement in energy community.

%\paragraph{Feedback on Achievements}
%In order to keep users motivated, we implement a set of achievements that are unlocked over time by using the app and performing the actions. For instance, users can get an achievement notification after they have performed a certain number of actions, accumulated a certain number of leaves individually or as part of some community, or after they have been using the app regularly for a certain period of time. Whenever a user unlocks an achievement in some category, s/he gets informed what is the next achievement in this category that s/he can reach. (E.g. ``Congrats! You completed your first $3$ actions! Your next goal is $5$ actions. Keep it up!'') Such achievements are based on the goal-setting \cite{Abrahamse2007265} and individual and collective feedback \cite{Abrahamse2013} approaches to behavioral changes in energy consumption.

\paragraph{Personalization and Localization}
A user has a personal profile and a household profile. We allow a user to customize the display name, preferred language (English, Italian or Swedish), and to provide information about the household composition, home type and size, major appliances, etc. The information is useful for personalized action suggestions, the comparison of similar households and individuals, and can be used for research purposes. A user (at the test sites) can link YouPower to the household's (DSO and sensors) energy data if an data account is provided. 
% 
In such cases, the app customizes its content to a user's test site: housing cooperatives content for the Swedish site and load-shifting content for the Italian site. They are discussed in the next two subsections. 


\paragraph{Design Evaluation}
The design was evaluated by peer reviews, a study with 24 participants 
in an environmentally-oriented event in Helsinki (\url{https://oscedays.org/helsinki/}), 
and a workshop with nine participants in the Italian test site. 
% 
In general, people liked the idea of receiving action suggestions. 
They like to see the impact of their actions and asked for easy to perform actions. 
The majority was interested in collaborative community actions, e.g., to save together and to donate for a common goal. Very few had interest in competition. 
% 
% Some people mentioned that the connection (or difference) between the \textit{Suggestions} and the \textit{Challenges} is hard to understand. Challenges were designed to express personal or community energy-goals, while suggestions are the means to achieve the goals. We later changed the name of this feature from \textit{Challenges} to \textit{Achievements}. 
% 
%Some people asked for the authority of the suggestions: it isn't clear where the tips come from. To what extent can they be trusted? Are they results of applied research? This is not marginal in a community of people who are quite familiar with energy-related issues. Having an explanation of how trustworthy these tips are would be very important. 
%%
%The meaning and value of the leaves are also another aspect that was questioned the most. 
%We thus decided to have an information page so that a number of such issues can be explained to the users. 
%In addition, users can send feedback to CIVIS through the app to keep the designers updated of users' concerns, questions, or other issues. 
% 
Many expressed the opinion that monetary savings are only somewhat important to them. They were also skeptical about how much money they can actually save. They instead showed interest to learn about energy saving strategies as they are driven by more intrinsic motives.
% 
Some participants think that the others (in their neighborhood or city) do not put the same effort in energy conservation as themselves do. The YouPower approach to display other people's actions may have the potential to motivate people seeing the others' efforts.
%
Some suggested that for those who do not have or are not comfortable with smart-phones, the app should be made available through a browser. 


\subsection{Housing Cooperatives}
\label{sect:brf}

This part of YouPower (see a screenshot in Figure~\ref{fig:brf}) is considered for households in the Stockholm test site. In Sweden, each apartment or house owner is a member of a housing cooperative that owns the property and annually elects a board that is in charge of the finances and maintenance of the property including making energy related decisions. 
Such a housing ownership concept exists in a number of EU and non-EU countries.
% 
\begin{figure}[t!]
\centering
\shadowimage[width=.8\linewidth]{img/brf.pdf}
\caption{Housing cooperatives part of YouPower}
\label{fig:brf}
\end{figure}
% 
Three main categories of features are designed for this part of YouPower: energy information about a user's own housing cooperative, energy information about other housing cooperatives, and support for communication between energy managers.
Expected primary users are energy managers and board members of housing cooperatives. Secondary users are ordinary cooperative members. 

Housing cooperative energy information includes comparative energy performance %(kWh /m$^{2}$)\footnote{We chose to use kWh in this case since cooperative managers have fairly good knowledge of energy units and the visualization is comparative.} 
and the cooperative's monthly and yearly energy use, divided into heating (including hot water) and facilities electricity. Energy actions that have been taken are listed in relation with energy consumptions. 
A user can see when different actions, such as energy information, optimisation or investments, were previously taken and see more details about the actions. By comparing the energy use with previous periods, the user can also see the impact of the actions.
% 
In the same way that the users can view information about their own cooperatives, they can also see the energy performance and energy actions taken by other cooperatives. This allows energy managers and others who are interested to e.g. explore the effect of a neighbouring cooperative's actions on their energy use and read about how they carried out an investment and which contractor was used. 
% 
To further support collaboration and knowledge exchange between housing cooperatives, there is a discussion group dedicated for energy managers. Within the group they have the possibility of creating discussion topics of their interests. In this way, the discussion of the occasional meetings with the local energy network can be extended to continue online.

\paragraph{Design Evaluation}

The design was evaluated with three energy managers and the feedback was incorporated in the design improvement of the application. The energy managers would primarily want to use the app to find housing cooperatives with similar challenges and see what actions they had taken. They also thought the app would be helpful for deciding which companies can be trusted based on what other housing cooperatives had done and what the effects were on the energy use. The energy managers doubted that other members in their cooperatives would be very interested in following the cooperative's energy use, but they thought the app might be useful for engaging members in specific questions. 

\subsection{Energy Data} 
\label{sect:load_shifting}

This part of YouPower (see screenshots in Figure~\ref{fig:viz}) is considered for households in the Italian test site.
The main focus is on leveraging load elasticity to maximize self-consumption of locally-installed Renewable Energy Sources (RESs); that is to support the shift of electrical consumption loads to match production peaks. A dynamic time-of-usage signal in combination with a scheme for energy donations is designed to assist this shift. Moreover, in order to make users aware of their consumption patterns, three different levels of energy consumption data are displayed in the app: household, appliances and community. 

At the household level, the current and historical consumptions are displayed. For users with production from renewable sources -- many households from the two Trentino test sites have roof-installed PV panels -- the app compares production and consumption levels with the aim to raise awareness of different prosumption patterns. % and to maximize self-consumption. 
% 
For users with installed smart plugs, the app visualizes consumption patterns at the single appliance level. This is meant to enable users to gain a deeper understanding of the relationship between their daily actions and the resulting energy consumption. 
%
At the community level, aggregated data is displayed about the energy balance of a community, (understood as the participants' aggregate consumptions done between consumption under green and red signals) -> ``this is not understandable''.
% The aggregated monthly consumption and production levels are reported. The share of self-consumption is also reported. 
%A single household's consumption level with is compared with the community average. This is meant to represent a powerful tool in helping users make sense of the actual consumption data, driving them towards the goal of a ``fair energy use''. At the same time, this can also be seen as a social mechanism, where community-level dynamics are used as a means to achieve a given goal, in this case increased energy efficiency at the single household level.

\begin{figure*}[t!]
\centering
%\frame{\includegraphics[width=0.7\linewidth]{img/brf.pdf}}
\shadowimage[width=.296\linewidth]{img/viz2.png}
\shadowimage[width=.296\linewidth]{img/viz1.png}
\shadowimage[width=.298\linewidth]{img/viz3.png}
\caption{Energy data part of YouPower}
\label{fig:viz}
\end{figure*}

\paragraph{Dynamic Time-of-Use Signals}

A model (?) is designed to predict the level of production from renewables in the subsequent 72 hours. It is used to generate time-of-use signals that are sent to users. The linear prediction model uses solar radiation data from both public and private sources\footnote{Meteotrentino \url{http://www.meteotrentino.it}, OpenWeatherMap \url{http://openweathermap.org}, Fondazione Edmund Mach \url{http://www.fmach.it} and US National Weather Service \url{http://nomads.ncep.noaa.gov}).} and historical data about the production of renewable plants provided by local energy consortia.
Estimating consumption patterns based on historical data, a matching engine is designed to forecast whether there will be surplus of local productions. If so, a favorable price is offered to users as an incentive for them to move flexible loads (e.g., dishwater and washing machine) to the perspective time intervals. 
In the app, the time-of-use signals are represented by green and red face emoticons in order to identify, respectively, when users can consume and when not. Signals are forecasted
for the following 72 hours, divided into three-hour intervals.

% The indication of the price for each time interval is accompanied by an icon, smiling if the price is below a given threshold (computed based on the historical price of energy in the previous two years) or presenting a sad expression otherwise. 

\paragraph{Energy Donation}
A user's contribution towards a better community energy load balance can bring about economic benefits for the electric consortia. Indeed, in both Trento test sites, local generations from RESs can cover the total consumptions at the aggregated yearly level. Yet, there are timing mismatch so that in order to serve demand peaks the consortia have to buy electricity from the national energy grid, while, at other times, local production exceeds demand, so that exceeding energy is sold. From an economic point of view, such transactions with the energy market are unfavorable, because
electricity surpluses are sold at a price that is lower than the national retail market price. At the same time, purchase of electricity from energy market is paid at higher price.
% 
The electrical consortia foresee benefits, in economic and infrastructural terms, from leveraging load shifts and are therefore willing to support an energy donation programme. 
%The donation mechanism co-designed with the local stakeholder foresees that the economic benefit for the consortia will be partially monetized in order to contribute to a project with a social goal. 
Indeed, users in the Trento test sites are part of such a donation programme, which is organized as participatory budgeting process, and from which they can opt-out. KwH consumed during peak of
production (\textit{i.e.} green face emoticon) contribute to the hoarding of collective KwH budget to be allocated to a beneficiary at the end of CIVIS trial period.
In each test site, participants are allowed to submit proposals, in form of a simple and concrete project ideas, in order to be awarded as final beneficiary of the hoarded KwH budget.
The app gives a description of the budgeting programme and provides information about the submitted proposals, so that users can be aware of what kind of beneficiary could benefit from the collective efforts. Furthermore, a simple chart shows how many KwH have been consumed in the green and red time zones.

%The donation mechanism, coupled with the dynamic ToU tariffing, is the key 'social' aspect foreseen for the Trento pilot site. It is based on the concept that the adoption of environmentally-friendly behaviors at the individual level can generate positive impacts at the social level ("social as a goal"). In this case impacts are on two different levels. First, in terms of reduced greenhouse gas emissions, as the local generation is totally based on RESs and hence carbon-neutral. Second, by contributing to the funding of a social project users can see a concrete, tangible effect of their action, fostering their motivation. 


\paragraph{Design Evaluation}
The design of the Energy Data part has been constructed and validated iteratively with users in Trento test sites. A preliminary inquiry of users' awareness about own energy behaviors and about energy as a collective matter was done through two focus groups. Later, two scenario-based workshops allowed to identify the main users' requirements for supporting the implementation of a load shifting intervention. Finally, two additional workshops were used to validate the proposed ideas, in terms of app's main functionalities, and to produce basic paper-based mock-ups, which CIVIS team used for defining the wire-frame for this part of YouPower.
%
What emerged clearly from the focus groups is that participants share a good sense of energy as a deeply local and collective matter, which is mostly due to the membership-based and cooperative nature of the electrical consortia. They highlighted the relevance of educating people about energy related matters at different levels: from education for the youngest generations in public schools to concrete tools for raising awareness about personal energy behaviors. More concretely, visualization of own consumption (and, for prosumers, production) profiles was considered as a primary and paramount requisite for attempting to shift loads. A predictive system emerged as another cornerstone for enabling participants' flexibility to plan their energy-related behaviors and pursue load shifts. However, comparisons and means for benchmarking consumptions also emerged as highly desirable features for placing people understanding of own energy behaviors into a broader context. The direction towards the participatory budget process for allocating bonus KwH emerged from participants' desire to have a transparent and participated process.
% During the focus group events, ideas about the possible functionalities were used as probes with the participants. During the workshops, practical activities were carried out with the participants in order to develop an idea of the most desirable features and then to receive evaluation feedback.
% 
% One issue that emerged was the interest in the production data and the ability to maximize self-consumption. The ideas of the dynamic ToU and the donation programme were also well received. 

