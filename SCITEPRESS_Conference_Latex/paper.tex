\documentclass[a4paper,twoside]{article}

\usepackage{epsfig}
\usepackage{subfigure}
\usepackage{calc}
\usepackage{amssymb}
\usepackage{amstext}
\usepackage{amsmath}
\usepackage{amsthm}
\usepackage{multicol}
\usepackage{pslatex}
\usepackage{apalike}
\usepackage{SCITEPRESS}     % Please add other packages that you may need BEFORE the SCITEPRESS.sty package.
\usepackage{url}

\subfigtopskip=0pt
\subfigcapskip=0pt
\subfigbottomskip=0pt

\begin{document}

\title{YouPower --  A Social App for User Engagement in Power Grids
%\subtitle{Preparation of Camera-Ready Contributions to SCITEPRESS Proceedings} 
}

\author{\authorname{First Author Name\sup{1}, Second Author Name\sup{1} and Third Author Name\sup{2}}
\affiliation{\sup{1}Institute of Problem Solving, XYZ University, My Street, MyTown, MyCountry}
\affiliation{\sup{2}Department of Computing, Main University, MySecondTown, MyCountry}
\email{\{f\_author, s\_author\}@ips.xyz.edu, t\_author@dc.mu.edu}
}

\keywords{Power Grid, Energy Community, Social Participation, Social App, YouPower}

\abstract{The abstract should summarize the contents of the paper and should contain at least 70 and at most 200 words. The text must be set to 9-point font size. \vspace*{2cm}}

\onecolumn \maketitle \normalsize \vfill

\section{\uppercase{Introduction}}
\label{sec:introduction}

\noindent 
This paper presents the design of YouPower, a social smart grid platform that is designed as a means to explore the potential and challenges of supporting social participation, awareness and engagement of power gird users (\url{https://app.civisproject.eu}). The goal of developing such a system is to make energy visible, to inform users' energy know-how, to promote
pro-environmental social norms, and to facilitate users in their day-to-day life to take energy-friendly actions together with online communities. 

X The idea of linking smart grids with (online) \textit{Social Network}s (SNs) as a joint R\&D topic has recently caught
much attention in media
\cite{Boslet2010,Chima2011,Erickson2012,Fang2013}. There are many
research efforts on either topics, but research on combining SNs with
smart grids has just started. A number of recent works propose frameworks or
approaches that interconnect smart meters (or smart homes) as SNs for
energy management and sharing \cite{Ciuciu2012,Steinheimer2012}. In
addition, Silva et. al.~\cite{Silva2012} conducted surveys to
understand user needs for energy services including SN
services. Several frameworks or simulation models for demand side
management and value-added web services with SN aspects have been
developed~\cite{Chatzidimitriou2013,De-Haan2011,Lei2012}. Others have
used simulation models to demonstrate the feasibility of social
coordination in supply and demand \cite{Skopik2014,Worm2013}.
% 
Our research interest expands on the related work in that it focuses
on smart grid user communities. The research is performed within the
framework of the EU FP7 CIVIS
project (\url{www.civisproject.eu}).

Project Context 


\begin{figure*}[t!]
\begin{center}\footnotesize
	\includegraphics[width=.7\textwidth]{img/civis_platform_overview.pdf}\\
	DSO (Distribution System Operators),  SSL (Secure Sockets Layer)
	\caption{YouPower system overview}\label{fig:platform}
\end{center}
\end{figure*}

\section{\uppercase{State of the Art}}

\noindent X Considering the analytical frames and the CIVIS use cases, we chose a set of platform features and translated those into three self-contained and composable parts to be included in the CIVIS (front-end) application (hereinafter abbreviated as CIVIS app)\footnote{A shortened list of YouPower app mock-ups can be found at \url{http://civis.tbm.tudelft.nl/mockups/}.}: 
%\begin{enumerate}
%\item \nameref{sect:tips}
%\item \nameref{sect:brf}
%\item \nameref{sect:load_shifting}
%\end{enumerate}



With peer review results and users' feedback on the design, adaptations and changes are made to suit user needs and to achieve the CIVIS research goal. 
In general, the application aims to enhance users' energy know-how through action suggestions that are implementable in everyday life, engage users in energy communities with understandable and actionable information and feedback, and facilitate community interaction and self-teaching by means of group discussions.
%

Given the time and resource constraints, the app can not be developed all-in-one cross-platform (for phones, tablets and computers). We chose to design the front-end as a mobile app. This means that the app design has layouts and user interactions that suit (small) phone screens. %The consideration is multi-fold. 
Western Europe has a large mobile phone internet user base\footnote{
Between 2013 and 2017, the penetration rate of mobile phone internet users among mobile phone users will rise from 49.0\% to 77.8\%. See more at:\url{ http://www.emarketer.com/Article/Nearly-Half-of-Western-Europeans-Will-Use-Mobile-Web-This-Year/1010510\#sthash.AaVfsqIU.dpuf}}. Many surveys show that mobile apps have advantages such as creating deeper user engagement, easy sharing, among others\footnote{\url{https://infomedia.com/blog/the-advantages-of-mobile-apps/}, \url{https://econsultancy.com/blog/62326-85-of-consumers-favour-apps-over-mobile-websites/}}. This makes mobile app a good choice given the goal of the CIVIS platform. Once developed, mobile apps can also be more easily transformed to web browser versions, while the reverse is more difficult. The back-end of the CIVIS platform will remain mostly the same independent of the front-end alternatives. 

\input{designConcept}


\section{\uppercase{Manuscript Preparation}}


\noindent {\bf Group 2.} Additionally, you may wish to copy and edit
the following 3 example files:
\begin{verbatim}
  - example.bib
  - example.tex
  - scitepress.eps
\end{verbatim}



\subsubsection{Tables}

\begin{table}[h]
\caption{This caption has more than one line so it has to be
justified.}\label{tab:example2} \centering
\begin{tabular}{|c|c|}
  \hline
  Example column 1 & Example column 2 \\
  \hline
  Example text 1 & Example text 2 \\
  \hline
\end{tabular}
\end{table}



%\begin{figure}[!h]
%  \vspace{-0.2cm}
%  \centering
%   {\epsfig{file = SCITEPRESS.eps, width = 5.5cm}}
%  \caption{This caption has more than one line so it has to be justified.}
%  \label{fig:example2}
%  \vspace{-0.1cm}
%\end{figure}


\subsubsection{Equations}

Equations should be placed on a separate line, numbered and
centered.\\The numbers accorded to equations should appear in
consecutive order inside each section or within the contribution,
with the number enclosed in brackets and justified to the right,
starting with the number 1.

Example:

\begin{equation}\label{eq1}
    a=b+c
\end{equation}

\subsubsection{Program Code}\label{subsubsec:program_code}

Program listing or program commands in text should be set in
typewriter form such as Courier New.

Example of a Computer Program in Pascal:

\begin{small}
\begin{verbatim}
 Begin
     Writeln('Hello World!!');
 End.
\end{verbatim}
\end{small}


\section{\uppercase{Conclusions}}
\label{sec:conclusion}


\section*{\uppercase{Acknowledgements}}

\noindent This research is funded by the EU FP7 CIVIS project.
%It is also partly funded by the NWO project RobuSmart
%(Increasing the Robustness of Smart Grids through distributed
%energy generation: a complex network approach), grant number
%647.000.001. The authors would like to thank Jukka K. Nurminen from Aalto Univ. and Jan M�uller from KIT for their
%comments.


\bibliographystyle{apalike}
{\small
\bibliography{bib}}


%\section*{\uppercase{Appendix}}
%
%\noindent If any, the appendix should appear directly after the
%references without numbering, and not on a new page. To do so please use the following command:
%\textit{$\backslash$section*\{APPENDIX\}}

\vfill
\end{document}

