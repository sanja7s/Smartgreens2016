\section{\uppercase{Introduction}}
\label{sec:introduction}

\noindent 
\textit{YouPower} is an open source platform\footnote{\url{http://www.civisproject.eu}, \url{https://app.civisproject.eu}, \url{https://github.com/CIVIS-project}. The packages are available for any
interested party to be reused, modified, and extended under the Apache v.2 License.} designed to explore the potential and challenges of supporting social participation, awareness and engagement of smart gird users for energy conservation and load shifting. Combining smart sensing and web technologies among others,
YouPower features a social smart grid application (developed as a hybrid mobile app) that can connect users to friends, families and local communities to learn and take energy actions that are relevant to them together. The app encourages an energy-friendly lifestyle and can be linked to users' energy consumption and production data for quasi real-time and historical prosumption information. 
% 
The goal of the project is to make energy more visible, to promote environmental and social values, to inform users' know-how about sustainable consumption, and to facilitate users to take energy conservation and load shifting actions in their everyday life together with local communities \cite{Huang2014,Huang2015c,Huang2016}. 
%  

Research topics related to merging the strength of Social Networks (SNs) with that of smart grid applications have caught much attention in recent years following the success of several popular SN platforms \cite{Boslet2010,Chima2011,Erickson2012,Fang2013,Huang2015}. 
% 
Some conducted surveys to understand user needs for energy services combining SNs \cite{Silva2012}. Some studied connecting smart meters (or smart homes) as SNs for
energy management and sharing \cite{Ciuciu2012,Steinheimer2012}. 
Simulation models are developed to study demand side management %and value-added web services 
taking into consideration SN aspects \cite{De-Haan2011,Lei2012,Chatzidimitriou2013} and to demonstrate the feasibility of
coordination in load balancing \cite{Worm2013,Skopik2014}. There are also works that visualize smart meter and appliance-level consumption data, and provide comparative feedback among households\cite{Petkov2011,Weiss2012,Dillahunt2014}.
% 
Our research interest expands on the related works, and places an emphasis on smart grid user communities and collective actions. 

The research is performed within the framework of the EU FP7 CIVIS project. It has test sites in Stockholm (Sweden) and Trento (Italy) with domestic energy consumers. 
% 
In Sweden, those who buy a home officially own the right to inhabit the estate and must join a corresponding \textit{housing cooperative} 
that owns and maintains the estates. 
%The test site in Stockholm is composed of \textbf{sixteen(?)} housing cooperatives, each of which has
The members of a cooperative annually elect a board that makes energy related decisions on behalf of the members. 
In the case of Trento test site, two local \textit{electricity consortia} produce and sell renewable (hydro and solar) energy to consortium members. Household rooftop PV panels are also common in this region.  The consortia are highly interested in load management to optimize the use of local renewables and reduce dependency on the national supply. 
These two types of communities are at the center of YouPower design. 
% 
The rest of this paper presents the design process of YouPower, gives an overview of the platform, and discusses in more detail its design concept.

